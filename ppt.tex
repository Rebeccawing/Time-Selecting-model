\documentclass{beamer}

\usepackage{graphicx}
\usepackage{booktabs}
\usetheme{Berlin}

\title{GFTD REPORT}

\subtitle{First Meeting}

\author{Shiying Cui}


\institute{Hyde Renaissance Capital Management Limited}


\date{Sept.11th, 2017}

\subject{GFTD Report}

\AtBeginSubsection[]
{
  \begin{frame}<beamer>{Outline}
    \tableofcontents[currentsection,currentsubsection]
  \end{frame}
}

\begin{document}

\begin{frame}
  \titlepage
\end{frame}

\begin{frame}{Outline}
  \tableofcontents
\end{frame}

\section{GFTD Strategy}

\subsection{Introduction}

\begin{frame}{Introduction}
GFTD is based on the Time Selecting System TD and aimed at selecting transaction time on market index.
  \begin{itemize}
  \item {
    The main logic is that we expect a reversal after a continuous rising or falling tendency.
    \pause
  }
  \item {
    There are three parameters in this strategy, named n$_1$,n$_2$,n$_3$ 
    \pause
  }
  \end{itemize}
\end{frame}

\subsection{Implements}

\begin{frame}{Implements}{Buy Trade Selection_start trade}
  \begin{itemize}
  \item {
    Closing price of day i is lower than that of day i-n$_1$.
	If this tendency continues for n$_2$ days, we start a count signal.
	\pause
  }
  \item {
    In the counting value process, if the three condition A,B and C are all satisfied, add count value by 1
  \begin{itemize}
  \item<3->{
    A. Closing price of day i is higher than the highest price of day i-2
  }
  \item<4->{
    B. The highest price of day i is higher than that of day i-1
  }
  \item<5->{
    C. Closing price of day i is higher than that of day i-1
  }
  \end{itemize}
  }
  \item<6->{
    If the count value reaches n$_3$, we start a buy trade.
  }
  \end{itemize}
\end{frame}

\begin{frame}{Implements}{Buy Trade Selection_stop trade}
  \begin{itemize}
  \item<1->{
    When to stop?
	\begin{itemize}
  \item<2->{
    A. If there forms a sell trade.
  }
  \item<3->{
    B. In the process of counting value, we record the lowest price of the market index.
	Once the closing price in buy trade is lower than the record price, we stop this buy trade.
  }
    \end{itemize}
  }
  \end{itemize}
\end{frame}

\begin{frame}{Implements}{Sell Trade Selection_start trade}
  \begin{itemize}
  \item {
    Closing price of day i is higher than that of day i-n$_1$.
	If this tendency continues for n$_2$ days, we start a count signal.
	\pause
  }
  \item {
    In the counting value process, if the three condition A,B and C are all satisfied, count value + 1
  \begin{itemize}
  \item<3->{
    A. Closing price of day i is lower than the lowest price of day i-2
  }
  \item<4->{
    B. The lowest price of day i is lower than that of day i-1
  }
  \item<5->{
    C. Closing price of day i is lower than that of day i-1
  }
  \end{itemize}
  }
  \item<6->{
    If the count value reaches n$_3$, we start a sell trade.
  }
  \end{itemize}
\end{frame}

\begin{frame}{Implements}{Sell Trade Selection_stop trade}
  \begin{itemize}
  \item<7->{
    When to stop?
	\begin{itemize}
  \item<8->{
    A. If there forms a buy trade.
  }
  \item<9->{
    B. In the process of counting value, we record the highest price of the market index.
	Once the closing price in buy trade is higher than the record price, we stop this sell trade.
  }
    \end{itemize}
  }
  \end{itemize}
\end{frame}

\section{Results and Analysis}

\subsection{Results}
\begin{frame}{Results}{Evaluation of this Model}
\small
  \begin{tabular}{*{2}{c}}
  \toprule
  Index name & Description\\
  \midrule
  Cumulative Return & Return Performance\\
  Annual Return & Return Performance\\
  Transaction Count & Reflect the Selection Times\\
  Win Count,Lose Count & Reflect the effectiveness\\
  Winning Percentage & Return Performance\\
  Average Return per Transaction & Return Performance\\
  Average Winning Return per Transaction & Single Transaction Return\\
  Average Losing Return per Transaction & Single Transaction Risk \\
  Odds & Winning Return/Losing Return\\
  Maximum Drawdown & Risk Performance\\
  Maximum Number of Consecutive Wins & Reflect return in a period\\
  Maximum Number of Consecutive Loses & Reflect risk in a period\\
  \bottomrule
  \end{tabular}
\end{frame}

\begin{frame}{Results}{Evaluation of this Model}
We get 11 indexes and three comparison graphs to evaluate the model.
\begin{figure}
\centering
\includegraphics[width=1.77in,height=1.50in]{cum_ret1.png}
\caption{Cumulative Return and Market Index}
\label{fig:graph}
\end{figure}
\end{frame}

\begin{frame}{Results}{Evaluation of this Model}
We get 11 indexes and three comparison graphs to evaluate the model.
\begin{figure}
\centering
\includegraphics[width=1.77in,height=1.50in]{cum_ret2.png}
\caption{Cumulative Return and Market Index}
\label{fig:graph}
\end{figure}
\end{frame}

\begin{frame}{Results}{Evaluation of this Model}
We get 11 indexes and three comparison graphs to evaluate the model.
\begin{figure}
\centering
\includegraphics[width=1.77in,height=1.5in]{max_drawdown.png}
\caption{Maximum Drawdown and Market Index}
\label{fig:graph}
\end{figure}
\end{frame}

\begin{frame}{Results}{Optimal Results}
\small
\begin{itemize}
  \item{
     In-sample: 2006-2014
	 Out-sample: 2015-2017
	 \pause
  }
  \item{
     Set an assess function to evaluate the strategy: Annual return+k*maximum drawdown.
	 k is a parameter that reflects the preference for high return or low risk.
	 \pause
  }
  \end{itemize}
  \begin{tabular}{*{4}{c}}
  \toprule
  Market Index & n$_1$ & n$_2$ & n$_3$\\
  \midrule
  SZ50 & 2 & 2 & 6\\
  HS300 & 3 & 2 & 6\\
  HS500 & 3 & 2 & 5\\
  HS800 & 3 & 2 & 5\\
  \bottomrule
  \end{tabular}
\end{frame}

\subsection{Analysis}
\begin{frame}{Analysis}
\begin{block}{Parameter Estimation}
Hope to find unbiased estimation for parameters n$_1$,n$_2$ and n$_3$
\pause
\end{block}
\begin{block}{Robustness}
The performance of parameters that near the optimal parameter combination is not very well.
\pause
\end{block}
\begin{block}{Sample selection}
Hope to explore different in-sample period such as 2009-2017 while out-sample is then 2006-2008.
\pause
\end{block}
\end{frame}

\section*{Summary}

\begin{frame}{Summary}
  \begin{itemize}
  \item
    This model can be edified flexibly and can satisfy arbitrary period.
    \pause
  \item
    This model is over fitting when finding the optimal parameters.
    \pause
  \end{itemize}
  \begin{itemize}
  \item
    Outlook
    \pause
    \begin{itemize}
    \item
      Build unbiased estimation of such parameters and develop more details in this model.
      \pause
    \item
      Build a more scientific and complicated assess function of the model.
      \pause
    \end{itemize}
  \end{itemize}
\end{frame}


\end{document}
